\chapter{SOLID}

\section{Analyse Single-Responsibility-Principle (SRP)}
[jeweils eine Klasse als positives und negatives Beispiel für SRP;  jeweils UML der Klasse und Beschreibung der Aufgabe bzw. der Aufgaben und möglicher Lösungsweg des Negativ-Beispiels (inkl. UML)]
\subsection{Positiv-Beispiel}
\subsection{Negativ-Beispiel}

\section{Analyse Open-Closed-Principle (OCP)}
[jeweils eine Klasse als positives und negatives Beispiel für OCP;  jeweils UML der Klasse und Analyse mit Begründung, warum das OCP erfüllt/nicht erfüllt wurde – falls erfüllt: warum hier sinnvoll/welches Problem gab es? Falls nicht erfüllt: wie könnte man es lösen (inkl. UML)?]
\subsection{Positiv-Beispiel}
\subsection{Negativ-Beispiel}

\section{Analyse Liskov-Substitution- (LSP), Interface-Segreggation- (ISP), Dependency-Inversion-Principle (DIP)}
\[jeweils eine Klasse als positives und negatives Beispiel für entweder LSP oder ISP oder DIP);  jeweils UML der Klasse und Begründung, warum man hier das Prinzip erfüllt/nicht erfüllt wird\] \\
\[Anm.: es darf nur ein Prinzip ausgewählt werden; es darf NICHT z.B. ein positives Beispiel für LSP und ein negatives Beispiel für ISP genommen werden\]
\subsection{Positiv-Beispiel}
\subsection{Negativ-Beispiel}
