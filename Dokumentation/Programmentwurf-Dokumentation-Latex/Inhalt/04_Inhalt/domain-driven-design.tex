\chapter{Domain Driven Design}
\section{Ubiquitous Language}
[4 Beispiele für die Ubiquitous Language; jeweils Bezeichung, Bedeutung und kurze Begründung, warum es zur Ubiquitous Language gehört]
\begin{table}[!ht]
\centering
\begin{tabular}{lll}
\textbf{Bezeichnung} & \textbf{Bedeutung} & \textbf{Begründung} \\
Klasse               & tbw                & tbw                 \\
Rasse                & tbw                & tbw                 \\
Equipment            & tbw                & tbw                 \\
Spell                & tbw                & tbw                
\end{tabular}
\end{table}

\FloatBarrier
\section{Entities}
[UML, Beschreibung und Begründung des Einsatzes einer Entity; falls keine Entity vorhanden: ausführliche Begründung, warum es keines geben kann/hier nicht sinnvoll ist]

\section{Value Objects}
[UML, Beschreibung und Begründung des Einsatzes eines Value Objects; falls kein Value Object vorhanden: ausführliche Begründung, warum es keines geben kann/hier nicht sinnvoll ist]

\section{Repositories}
[UML, Beschreibung und Begründung des Einsatzes eines Repositories; falls kein Repository vorhanden: ausführliche Begründung, warum es keines geben kann/hier nicht sinnvoll ist]

\section{Aggregates}
[UML, Beschreibung und Begründung des Einsatzes eines Aggregates; falls kein Aggregate vorhanden: ausführliche Begründung, warum es keines geben kann/hier nicht sinnvoll ist]