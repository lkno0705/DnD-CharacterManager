\chapter{Unit Tests}
\section{10 Unit Tests}
Tabelle irgendwo in diesem Kapitel. Latex packt die dahin wo es das für richtig hält
\afterpage{%
	\clearpage% Flush earlier floats (otherwise order might not be correct)
	\thispagestyle{empty}% empty page style (?)
	\begin{landscape}% Landscape page
		\centering % Center table
		\begin{adjustbox}{center,caption={Evaluation der Datensätze},float=table,label={tab:Datasets}}
			\centering
			\label{tab:Datasets}
			\small
			\setlength{\tabcolsep}{3pt}
	\begin{tabular}{l|l}
		Unit Test                                & Beschreibung                                                                       \\ \hline
		RPGCharacterTest\#getAC()                & Es wird geprüft ob die Armor Class (AC) des Charackters korrekt berechnet wird.    \\ \hline
		RPGCharacterTest\#getSavingThrows()      & Es wird geprüft ob die SavingThrows Boni des Charackters korrekt berechnet werden. \\ \hline
		RPGCharacterTest\#getSkills()            & Es wird geprüft ob die Skill Boni des Charackters korrekt berechnet werden.        \\ \hline
		DeathSavesTest\#getFailures()            & Prüft ob die Anzahl an Fehlschlägen korrekt berechnet wird.                        \\ \hline
		DeathSavesTest\#getSuccesses()           & Prüft ob die Anzahl an Erfolgreichen Death Saves korrekt berechnet wird            \\ \hline
		DiceRollServiceTest\#rollInitiative()    & Prüft ob das Ergebnis des Initiative Wurfs korrekt berechnet wird                  \\ \hline
		DiceRollServiceTest\#attack()            & Prüft ob der Damage beim Ausführen einer Attacke korrekt berechnet wird            \\ \hline
		DiceRollServiceTest\#rollSkill()         & Prüft ob das Ergebnis eines Skill Rolls korrekt berechnet wird                     \\ \hline
		DiceRollServicetest\#rollSavingThrow()   & Prüft ob das Ergebnis eines SavingThrows korrekt berechnet wird                    \\ \hline
		CharacterServiceTest\#displayCharacter() & Prüft ob der String eines Charackters korrekt zusammengebaut wird.                
	\end{tabular}
       % \caption{Evaluation der Datensätze}
\end{adjustbox}
% Add 'table' caption
\end{landscape}
\clearpage% Flush page
}
\section{ATRIP: Automatic}
[Begründung/Erläuterung, wie ‘Automatic’ realisiert wurde]

\section{ATRIP: Thorough}
[jeweils 1 positives und negatives Beispiel zu ‘Thorough’; jeweils Code-Beispiel, Analyse und Begründung, was professionell/nicht professionell ist]

\section{ATRIP: Professional}
[jeweils 1 positives und negatives Beispiel zu ‘Professional’; jeweils Code-Beispiel, Analyse und Begründung, was professionell/nicht professionell ist]

\section{Code Coverage}
[Code Coverage im Projekt analysieren und begründen]

\section{Fakes und Mocks}
[Analyse und Begründung des Einsatzes von 2 Fake/Mock-Objekten; zusätzlich jeweils UML Diagramm der Klasse]