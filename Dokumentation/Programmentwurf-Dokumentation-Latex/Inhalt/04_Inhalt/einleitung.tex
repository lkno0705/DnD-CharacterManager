\chapter{Einführung}

\section{Übersicht über die Applikation}
Die Abgegebene Applikation, \enquote{DnD-CharacterManager}, ermöglicht das erstellen und verwalten von Dungeons \& Dragons Charactären nach dem 5e Regelwerk. Somit soll sie den bekannten Papiercharacterbogen ablösen und durch eine digitale Version ersetzen. Neben dem persistenten halten von Daten nach dem In-Memory Prinzip moderner Datenbanken, annulliert sie ausserdem das anstrengende Kopfrechnen bei verschiedenen Checks und stellt einen Character Creation wizard bereit, der den Nutzer durch den Characktererstellungsprozess führt.

\section{Wie startet man die Applikation?}
Die Applikation kann man starten, in dem man die bereitgestellte *.jar Datei in einem Terminal emulator mit dem Befehl
\texttt{java -jar DnDCharacterManager-jar-with-dependencies.jar} ausführt. Vorraussetzung dafür ist eine valide funktionierende Java installation. Das projekt wurde mittels des \texttt{openjdk-18} und einem \enquote{language level} von 17 erstellt. Bitte verwenden sie zum ausführen eine ähnliche Version des jdks, da es ansonsten zu Problemen und inkompatibilität kommen kann.

\section{Wie testet man die Applikation?}
Die Unit Tests der Applikation können über Maven mit Maven test, package und install ausgeführt werden. Die Applikation selbst lässt sich in der Kommandozeile bedienen.
